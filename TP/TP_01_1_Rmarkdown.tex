% Options for packages loaded elsewhere
\PassOptionsToPackage{unicode}{hyperref}
\PassOptionsToPackage{hyphens}{url}
%
\documentclass[
]{article}
\usepackage{amsmath,amssymb}
\usepackage{lmodern}
\usepackage{iftex}
\ifPDFTeX
  \usepackage[T1]{fontenc}
  \usepackage[utf8]{inputenc}
  \usepackage{textcomp} % provide euro and other symbols
\else % if luatex or xetex
  \usepackage{unicode-math}
  \defaultfontfeatures{Scale=MatchLowercase}
  \defaultfontfeatures[\rmfamily]{Ligatures=TeX,Scale=1}
\fi
% Use upquote if available, for straight quotes in verbatim environments
\IfFileExists{upquote.sty}{\usepackage{upquote}}{}
\IfFileExists{microtype.sty}{% use microtype if available
  \usepackage[]{microtype}
  \UseMicrotypeSet[protrusion]{basicmath} % disable protrusion for tt fonts
}{}
\makeatletter
\@ifundefined{KOMAClassName}{% if non-KOMA class
  \IfFileExists{parskip.sty}{%
    \usepackage{parskip}
  }{% else
    \setlength{\parindent}{0pt}
    \setlength{\parskip}{6pt plus 2pt minus 1pt}}
}{% if KOMA class
  \KOMAoptions{parskip=half}}
\makeatother
\usepackage{xcolor}
\IfFileExists{xurl.sty}{\usepackage{xurl}}{} % add URL line breaks if available
\IfFileExists{bookmark.sty}{\usepackage{bookmark}}{\usepackage{hyperref}}
\hypersetup{
  pdftitle={TP 1.1 - Introduction à R markdown},
  pdfauthor={Paul Bastide},
  hidelinks,
  pdfcreator={LaTeX via pandoc}}
\urlstyle{same} % disable monospaced font for URLs
\usepackage[margin=1in]{geometry}
\usepackage{color}
\usepackage{fancyvrb}
\newcommand{\VerbBar}{|}
\newcommand{\VERB}{\Verb[commandchars=\\\{\}]}
\DefineVerbatimEnvironment{Highlighting}{Verbatim}{commandchars=\\\{\}}
% Add ',fontsize=\small' for more characters per line
\usepackage{framed}
\definecolor{shadecolor}{RGB}{248,248,248}
\newenvironment{Shaded}{\begin{snugshade}}{\end{snugshade}}
\newcommand{\AlertTok}[1]{\textcolor[rgb]{0.94,0.16,0.16}{#1}}
\newcommand{\AnnotationTok}[1]{\textcolor[rgb]{0.56,0.35,0.01}{\textbf{\textit{#1}}}}
\newcommand{\AttributeTok}[1]{\textcolor[rgb]{0.77,0.63,0.00}{#1}}
\newcommand{\BaseNTok}[1]{\textcolor[rgb]{0.00,0.00,0.81}{#1}}
\newcommand{\BuiltInTok}[1]{#1}
\newcommand{\CharTok}[1]{\textcolor[rgb]{0.31,0.60,0.02}{#1}}
\newcommand{\CommentTok}[1]{\textcolor[rgb]{0.56,0.35,0.01}{\textit{#1}}}
\newcommand{\CommentVarTok}[1]{\textcolor[rgb]{0.56,0.35,0.01}{\textbf{\textit{#1}}}}
\newcommand{\ConstantTok}[1]{\textcolor[rgb]{0.00,0.00,0.00}{#1}}
\newcommand{\ControlFlowTok}[1]{\textcolor[rgb]{0.13,0.29,0.53}{\textbf{#1}}}
\newcommand{\DataTypeTok}[1]{\textcolor[rgb]{0.13,0.29,0.53}{#1}}
\newcommand{\DecValTok}[1]{\textcolor[rgb]{0.00,0.00,0.81}{#1}}
\newcommand{\DocumentationTok}[1]{\textcolor[rgb]{0.56,0.35,0.01}{\textbf{\textit{#1}}}}
\newcommand{\ErrorTok}[1]{\textcolor[rgb]{0.64,0.00,0.00}{\textbf{#1}}}
\newcommand{\ExtensionTok}[1]{#1}
\newcommand{\FloatTok}[1]{\textcolor[rgb]{0.00,0.00,0.81}{#1}}
\newcommand{\FunctionTok}[1]{\textcolor[rgb]{0.00,0.00,0.00}{#1}}
\newcommand{\ImportTok}[1]{#1}
\newcommand{\InformationTok}[1]{\textcolor[rgb]{0.56,0.35,0.01}{\textbf{\textit{#1}}}}
\newcommand{\KeywordTok}[1]{\textcolor[rgb]{0.13,0.29,0.53}{\textbf{#1}}}
\newcommand{\NormalTok}[1]{#1}
\newcommand{\OperatorTok}[1]{\textcolor[rgb]{0.81,0.36,0.00}{\textbf{#1}}}
\newcommand{\OtherTok}[1]{\textcolor[rgb]{0.56,0.35,0.01}{#1}}
\newcommand{\PreprocessorTok}[1]{\textcolor[rgb]{0.56,0.35,0.01}{\textit{#1}}}
\newcommand{\RegionMarkerTok}[1]{#1}
\newcommand{\SpecialCharTok}[1]{\textcolor[rgb]{0.00,0.00,0.00}{#1}}
\newcommand{\SpecialStringTok}[1]{\textcolor[rgb]{0.31,0.60,0.02}{#1}}
\newcommand{\StringTok}[1]{\textcolor[rgb]{0.31,0.60,0.02}{#1}}
\newcommand{\VariableTok}[1]{\textcolor[rgb]{0.00,0.00,0.00}{#1}}
\newcommand{\VerbatimStringTok}[1]{\textcolor[rgb]{0.31,0.60,0.02}{#1}}
\newcommand{\WarningTok}[1]{\textcolor[rgb]{0.56,0.35,0.01}{\textbf{\textit{#1}}}}
\usepackage{graphicx}
\makeatletter
\def\maxwidth{\ifdim\Gin@nat@width>\linewidth\linewidth\else\Gin@nat@width\fi}
\def\maxheight{\ifdim\Gin@nat@height>\textheight\textheight\else\Gin@nat@height\fi}
\makeatother
% Scale images if necessary, so that they will not overflow the page
% margins by default, and it is still possible to overwrite the defaults
% using explicit options in \includegraphics[width, height, ...]{}
\setkeys{Gin}{width=\maxwidth,height=\maxheight,keepaspectratio}
% Set default figure placement to htbp
\makeatletter
\def\fps@figure{htbp}
\makeatother
\setlength{\emergencystretch}{3em} % prevent overfull lines
\providecommand{\tightlist}{%
  \setlength{\itemsep}{0pt}\setlength{\parskip}{0pt}}
\setcounter{secnumdepth}{-\maxdimen} % remove section numbering
\ifLuaTeX
  \usepackage{selnolig}  % disable illegal ligatures
\fi

\title{TP 1.1 - Introduction à \texttt{R\ markdown}}
\author{Paul Bastide}
\date{02/04/2024}

\begin{document}
\maketitle

\hypertarget{r-markdown}{%
\subsection{\texorpdfstring{\texttt{R\ Markdown}}{R Markdown}}\label{r-markdown}}

\texttt{R\ Markdown} permet de créer des documents dynamiques, qui
incluent du texte mis en forme, des équations, et du code \texttt{R}.

Cet outil est très utile pour écrire des rapports techniques. Il permet,
dans un seul document, d'exposer le contexte du problème, la méthode de
résolution, et les résultats de l'analyse.

De nombreuses ressources sont disponibles en ligne. Voir par exemple
\href{https://rmarkdown.rstudio.com/lesson-1.html}{l'introduction
officielle} à \texttt{R\ markdown}, ainsi que la
\href{https://raw.githubusercontent.com/rstudio/cheatsheets/master/rmarkdown-2.0.pdf}{fiche
synthétique}.

\hypertarget{installation}{%
\subsection{Installation}\label{installation}}

Pour utiliser \texttt{R\ markdown}, il suffit d'installer la librairie
associée, en tapant la commande suivante dans la console:

\begin{Shaded}
\begin{Highlighting}[]
\FunctionTok{install.packages}\NormalTok{(}\StringTok{"rmarkdown"}\NormalTok{)}
\end{Highlighting}
\end{Shaded}

\hypertarget{cruxe9ation-dun-document}{%
\subsection{Création d'un document}\label{cruxe9ation-dun-document}}

Une fois l'installation effectuée, le plus simple est de créer un
nouveau document en utilisant l'interface de \texttt{RStudio}.

Sélectionner le menu:

\begin{verbatim}
File > New File > R markdown...
\end{verbatim}

Dans l'onglet \texttt{Document} (sélectionné par défaut), vous pouvez
saisir le titre et la, le ou les auteur·e·s du document.

Trois formats sont proposés:

\begin{itemize}
\item
  \texttt{HTML}: permet une mise en forme dynamique, utile pour les
  sites internet.
\item
  \texttt{PDF}: permet une mise en forme fixe, utile pour un rendu
  ``officiel'' ou papier.
\item
  \texttt{Word}: permet de générer des documents éditables, qui peuvent
  ensuite être partagés avec des collaborateurs ou collaboratrices.
\end{itemize}

Pour cette séance, on choisi le format \texttt{HTML}, le plus simple
pour la mise en page.

\hypertarget{premier-document}{%
\subsection{Premier document}\label{premier-document}}

Une fois le document créé, \texttt{RStudio} propose un contenu
``didactique'' par défaut.

Vous pouvez compiler ce document en cliquant sur la commande
\texttt{Knit} en haut à gauche, à côté d'une pelote de laine (en
anglais, ``to knit'' signifie ``tricoter'').

Si tout fonctionne bien, \texttt{RStudio} va générer un document
\texttt{HTML}, qu'il ouvre dans une nouvelle fenêtre.

Vous pouvez étudier ce premier document, et voir comment la source (le
\texttt{.Rmd}) influe sur la sortie (le \texttt{.html}).

\hypertarget{code-r-et-graphiques}{%
\subsection{\texorpdfstring{Code \texttt{R} et
graphiques}{Code R et graphiques}}\label{code-r-et-graphiques}}

Supposons que l'on mène l'expérience suivante.\\
Une urne contient 300 boules, 100 rouges, 100 bleues et 100 vertes.\\
On tire une boule au hasard, on note sa couleur, et on la met de côté.\\
On reproduit cette expérience 60 fois.

Avec \texttt{R}, on peut simuler cette expérience aléatoire de la façon
suivante.

\begin{Shaded}
\begin{Highlighting}[]
\NormalTok{urne }\OtherTok{\textless{}{-}} \FunctionTok{c}\NormalTok{(}\FunctionTok{rep}\NormalTok{(}\StringTok{"rouge"}\NormalTok{, }\DecValTok{100}\NormalTok{),           }\DocumentationTok{\#\# 100 boules rouges}
          \FunctionTok{rep}\NormalTok{(}\StringTok{"bleue"}\NormalTok{, }\DecValTok{100}\NormalTok{),           }\DocumentationTok{\#\# 100 boules bleues}
          \FunctionTok{rep}\NormalTok{(}\StringTok{"verte"}\NormalTok{, }\DecValTok{100}\NormalTok{))           }\DocumentationTok{\#\# 100 boules vertes }

\NormalTok{n\_exp }\OtherTok{\textless{}{-}} \DecValTok{60}                            \DocumentationTok{\#\# Nombre de fois où je fais l\textquotesingle{}expérience}

\NormalTok{echantillon }\OtherTok{\textless{}{-}} \FunctionTok{sample}\NormalTok{(urne,            }\DocumentationTok{\#\# échantillonne les boules}
\NormalTok{                      n\_exp,           }\DocumentationTok{\#\# n\_exp fois}
                      \AttributeTok{replace =} \ConstantTok{FALSE}\NormalTok{) }\DocumentationTok{\#\# sans remise}
\end{Highlighting}
\end{Shaded}

On peut ensuite afficher un résumé de l'expérience (nombre de fois où
chaque boule a été tirée).

\begin{Shaded}
\begin{Highlighting}[]
\NormalTok{res }\OtherTok{\textless{}{-}} \FunctionTok{table}\NormalTok{(echantillon)             }\DocumentationTok{\#\# Résumé de l\textquotesingle{}échantillon}
\NormalTok{res}
\end{Highlighting}
\end{Shaded}

\begin{verbatim}
## echantillon
## bleue rouge verte 
##    24    18    18
\end{verbatim}

Et tracer ce résultat.

\begin{Shaded}
\begin{Highlighting}[]
\FunctionTok{barplot}\NormalTok{(res)                          }\DocumentationTok{\#\# Diagramme en bar}
\end{Highlighting}
\end{Shaded}

\includegraphics{TP_01_1_Rmarkdown_files/figure-latex/boules-resultats-tracé-1.pdf}

Il est aussi possible d'utiliser les résultats dans le texte.

Par exemple, sur les 60 expériences, on a tiré ici 24 fois la boule
bleue.

\hypertarget{exercice}{%
\subsection{Exercice}\label{exercice}}

\begin{quote}
\textbf{À rendre}: votre rapport \emph{compilé} sous format
\texttt{html}.
\end{quote}

Rédigez un document \texttt{R\ markdown} qui réponde au problème exposé
ci-dessous. Le document doit être auto-suffisant, et exposer clairement
le problème, les analyses, et les conclusions.

Il contiendra des sections, des mots mis en valeur en \textbf{gras} et
en \emph{italique}, et du \texttt{code} dans le texte.

Il contiendra également des blocs de code \texttt{R}, qui serviront à
répondre au problème.

Il pourra contenir des équations \(\LaTeX\), et des listes, numérotées
ou non.

\begin{center}\rule{0.5\linewidth}{0.5pt}\end{center}

\begin{quote}
On pêche des poissons dans le Lez.

On suppose qu'il y a en tout \(10\,000\) poissons dans le Lez, dont
\(2\,000\) rouges, \(3\,000\) verts et \(5\,000\) bleus (le Lez est très
pollué). On suppose cependant que l'on n'a pas accès à cette information
(on ne sait pas combien il y a de poissons en tout, ni combien de chaque
couleur).

On se pose la question : ``Quelle est la proportion de poissons rouges,
verts et bleus dans le Lez ?''

On pêche \(100\) poissons, que l'on garde pour les exposer en aquarium,
et l'on note leur couleur.

Décrivez l'expérience statistique : question posée, individus,
population, échantillon, taille, type de variable mesurée. La variable
a-t-elle bien des modalités incompatibles, exhaustives et sans ambiguïté
?

Simulez les données correspondant à cette expérience, en utilisant la
fonction \texttt{sample}.

Calculer la fréquence empirique de chaque couleur de poissons en
utilisant la fonction \texttt{table}. Quelle est la proportion estimée
de poissons rouges, verts et bleus ? Ce résultat vous surprend-il ?

Tracer ces fréquences empiriques en utilisant la fonction
\texttt{barplot}.

Que se passe-t-il lorsque vous compilez votre document plusieurs fois de
suite ? Les résultats numériques changent-ils ? Est-ce normal ? Utilisez
la fonction \texttt{set.seed} pour produire un document reproductible.

On suppose que, par manque de budget, on ne peut pêcher que \(10\)
poissons. Quels résultats obtenez-vous ? L'estimation des proportions
est-elle bonne dans ce cas ?

On suppose maintenant que, grâce à une collecte de fonds en ligne, on
peut pêcher \(1\,000\) poissons. Est-ce que cela améliore l'estimation ?
\end{quote}

\begin{center}\rule{0.5\linewidth}{0.5pt}\end{center}

\end{document}
